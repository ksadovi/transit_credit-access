\documentclass{beamer}
\usepackage{kyradefaults}

% Custom UChicago theme color (Maroon)
\definecolor{uchicagored}{RGB}{128,0,0}

\usepackage[style=chicago-authordate,backend=biber]{biblatex}
\addbibresource{references.bib}

\usepackage{tikz}
\usetikzlibrary{trees}
\usepackage{graphicx}
\usepackage{hyperref}
\usepackage{subcaption}
\usepackage{amsmath,amssymb}

% Theme setup
\usetheme{Madrid}
\usecolortheme[named=uchicagored]{structure}

% Footer setup
\setbeamertemplate{footline}{
  \leavevmode%
  \vspace{1ex}
  \hbox to \paperwidth{
    \hspace{1em}
    \usebeamerfont{author in head/foot}Kyra Sadovi (UChicago Harris)%
    \hfill
    \usebeamerfont{date in head/foot}\textit{All Aboard? New Rail Stations and Labor Markets}%
    \hfill
    \usebeamerfont{date in head/foot}\insertshortdate%
    \hspace{1em}
    \usebeamerfont{page number in head/foot}%
    \usebeamercolor[fg]{page number in head/foot}%
    \insertframenumber{} / \inserttotalframenumber%
    \hspace{1em}%
    \vspace{1em}%
  }%
  \vspace{0.5ex}
}

% Remove navigation symbols
\setbeamertemplate{navigation symbols}{}

% Title info
\title[New Rail Stations and Labor Markets]{All Aboard? Causal Evidence on Labor-Market Effects of New Rail Stations}
\author{Kyra Sadovi}
\author{Kyra Sadovi}
\institute{University of Chicago Harris School of Public Policy \\
\href{mailto:ksadovi@uchicago.edu}{ksadovi@uchicago.edu}}
\date{\today}

\begin{document}

% Title slide
\begin{frame}[plain]
  \titlepage
\end{frame}

% Motivation / Research Question
\begin{frame}{Motivation and Research Question}
\textbf{Goal:} Estimate how increased physical access to public transit affects nearby residents' labor-market outcomes.

\bigskip
\textbf{Outcomes:}
\begin{itemize}
    \item Worker flows (job retention, job-to-job moves)
    \item Income
\end{itemize}

\bigskip
\textbf{Research Question:}
\begin{itemize}
    \item What is the causal effect of gaining access to a new rail station on local labor-market outcomes?
\end{itemize}
\end{frame}

% Mechanisms
\begin{frame}{Conceptual Mechanisms}
\small
I posit three main channels linking new rail access to labor-market outcomes:
\begin{enumerate}
    \item \textbf{Job retention:} More reliable commuting makes it easier to keep existing jobs.
    \item \textbf{Job search and matching:} Expanded effective job-search radius increases access to higher-paying or better-matched jobs.
    \item \textbf{Networks and referrals:} Improved access to more locations broadens social and professional networks, increasing referrals and information flows.
\end{enumerate}

\bigskip
\textbf{Empirical challenge:} Separating these effects from underlying trends in dense, already well-connected areas.
\end{frame}

% Identification challenges overview
\begin{frame}{Identification Challenges}
\small
Three main identification problems:
\begin{enumerate}
    \item \textbf{Anticipation effects:}
    \begin{itemize}
        \item Households and firms respond when projects are announced, well before stations open.
        \item Evidence from the Chicago Orange Line shows capitalization in house prices up to 6 years before opening \parencite{mcmillen_reaction_2004}.
    \end{itemize}
    \item \textbf{Resident sorting:}
    \begin{itemize}
        \item High-income or more attached workers may choose more connected neighborhoods.
    \end{itemize}
    \item \textbf{Station location choice:}
    \begin{itemize}
        \item Planners site stations in dense or “important” locations (\`a la airport rail lines in \textcite{tyndall_airports}).
    \end{itemize}
\end{enumerate}
\end{frame}

% Anticipation effects
\begin{frame}{Anticipation Effects}
\small
\textbf{Issue:} Long construction timelines $\Rightarrow$ behavior and prices adjust before physical access improves.

\medskip
\textcite{mcmillen_reaction_2004} (Chicago Orange Line):
\begin{itemize}
    \item Anticipated benefits start capitalizing into house prices up to 6 years before completion.
\end{itemize}

\medskip
\textbf{My approach:}
\begin{itemize}
    \item Construct a \textbf{project-time index} using EIS data: $t=0$ is scheduled opening; $t<0$ denotes pre-opening period.
    \item Compare projects at the same relative time (same $t$) but with different realized status: open vs.\ delayed.
\end{itemize}
\end{frame}

% Non-random exposure & BH
\begin{frame}{Non-Random Exposure and Resident Sorting}
\small
\textbf{Resident sorting:}
\begin{itemize}
    \item Households and firms do not choose neighborhoods at random.
    \item Areas slated for transit are often already more central, dense, or high-income.
\end{itemize}

\medskip
\textbf{Consequence:}
\begin{itemize}
    \item Even if delays are “as good as random,” exposure to new stations is not.
    \item Standard DiD or IV may conflate timing shocks with underlying spatial differences.
\end{itemize}

\medskip
\textbf{Solution: \citeauthor{borusyak_nonrandom_2023} recentered instrument}
% I want to add a button here that goes to an Appendix slide describing the BH approach and compare and contrast what I'm doing 
\begin{itemize}
    \item Treat \textbf{construction delays} as exogenous shocks $g$.
    \item Map shocks and baseline geography into an exposure measure $z_{ct}(g_t; w)$.
    \item Subtract off the \emph{expected} exposure $\mu_{ct} = \mathbb{E}[z_{ct} \mid w]$ to obtain a recentered instrument $\tilde{z}_{ct} = z_{ct} - \mu_{ct}$. More on this in a few slides. 
\end{itemize}
\end{frame}

% Station location choice
\begin{frame}{Endogenous Station Location Choice}
\small
\textbf{Planner-side sorting:}
\begin{itemize}
    \item New stations are sited in already-important locations:
    \begin{itemize}
        \item Dense residential neighborhoods
        \item Central business districts 
        \item Retail hubs
        \item Airports and major transit nexuses \parencite{tyndall_airports}
    \end{itemize}
\end{itemize}

\medskip
\textbf{My design:}
\begin{itemize}
    \item Restrict attention to neighborhoods already selected to receive a station. This is built into the delay structure.
    \item Vary only the \emph{timing} of opening via delays, not the siting decision.
    \item Causal parameter is an ATT: effect of earlier vs.\ later access among already-chosen locations.
\end{itemize}

\medskip
\textbf{Possible refinement:}
\begin{itemize}
    \item Classify stations by type (CBD vs.\ residential vs.\ airport) and control for these in robustness checks.
\end{itemize}
\end{frame}

% Data: LODES
\begin{frame}{Data: Worker Flows (LODES)}
\small
\textbf{LODES:}
\begin{itemize}
    \item Origin–Destination Employment Statistics (LODES): worker home and work tracts.
    \item Public version: Census tract-level OD flows.
\end{itemize}

\medskip
\textbf{Use in this project:}
\begin{itemize}
    \item Construct tract-level measures of:
    \begin{itemize}
        \item Worker inflows and outflows
        \item Job retention and job-to-job transitions
        \item Income proxies (where available)
    \end{itemize}
    \item Preliminary analysis: tract-level regressions.
    \item Goal: access restricted individual-level LODES to:
    \begin{itemize}
        \item Compute door-to-door travel times at the individual level.
        \item Implement the CMA specification more credibly.
    \end{itemize}
\end{itemize}
\end{frame}

% Data: EIS / delays
\begin{frame}{Data: Measuring Delays (EIS and Project Databases)}
\small
\textbf{Transit Costs Project:}
\begin{itemize}
    \item Base list of major US transit projects over the last $\sim$20 years.
\end{itemize}

\medskip
\textbf{EPA EIS database:}
\begin{itemize}
    \item Environmental Impact Statements provide:
    \begin{itemize}
        \item Announcement / finalization dates
        \item Intended opening dates
        \item Realized opening dates (where available)
    \end{itemize}
\end{itemize}

\medskip
\textbf{Time index:}
\begin{itemize}
    \item $t=0$: scheduled opening date (from EIS).
    \item $t<0$: pre-opening period (announcement to planned opening).
    \item $D_s = R_s - P_s$: delay variable used to construct shocks and relative-time comparisons.
\end{itemize}
\end{frame}

% Data: Walking time / r5r + figure
\begin{frame}{Data: Walking Time and Network Access (r5r)}
\small
\textbf{r5r package \parencite{r5r}:}
\begin{itemize}
    \item Builds multimodal routing graphs (walk + bus + rail) from GTFS and OpenStreetMap.
    \item I use only the \emph{pre-period} pedestrian network to keep exposure predetermined.
\end{itemize}

\medskip
\textbf{Implementation:}
\begin{itemize}
    \item For each tract centroid:
    \begin{itemize}
        \item Compute 5, 10, 15, and 30-minute walking isochrones.
        \item Intersect polygons with station locations to define bands $\mathcal{B}_1$–$\mathcal{B}_3$ and $\mathcal{S}_c(30)$.
    \end{itemize}
    \item Result: tract–station distance matrix $d^{\text{walk}}_{cs}$ and, later, travel-time matrices $\tau_{ck}(t; g_t)$.
\end{itemize}

\medskip
\textbf{Motivation for individual data:}
\begin{itemize}
    \item Tract centroids justify the binned exposure measure.
    \item Individual LODES data would allow a fully continuous CMA measure at the person level.
\end{itemize}
\end{frame}

\begin{frame}{Example: Isochrones and Worker Flows (Proof of Concept)}
\centering
\begin{subfigure}[b]{0.45\textwidth}
    \centering
    \includegraphics[width=\textwidth]{3_output/2_figures/1_maps/1_station_geographies/WMATA.png}
    % \caption{Walking Isochrones (Example)}
    \label{fig:dmv_walkingmap_beamer}
\end{subfigure}
\begin{subfigure}[b]{0.45\textwidth}
    \centering
    \includegraphics[width=\textwidth]{3_output/dmv_map.png}
    % \caption{Worker Outflows (Cross-Section)}
    \label{fig:outflows_beamer}
\end{subfigure}

\bigskip
\small
These maps illustrate how \texttt{r5r}-based isochrones and LODES worker flows can be combined to study how changes in rail access reshape local commuting patterns.
\end{frame}

% Model setup overview
\begin{frame}{Model Setup: Overview}
\small
\textbf{Units and shocks:}
\begin{itemize}
    \item Units: Census tracts $c$ (public LODES data).
    \item Shocks: station-level delays $D_s = R_s - P_s$.
    \item Shock vector: $g_t = \{D_s\}_s \Rightarrow O_{st}(g_t) = \mathbf{1}\{t \ge P_s + D_s\}$.
\end{itemize}

\medskip
\textbf{Exposure:}
\begin{itemize}
    \item Pre-period walking-time access to stations (via \texttt{r5r}).
    \item Baseline network centrality and employment/population.
\end{itemize}

\medskip
\textbf{Approach:}
\begin{itemize}
    \item Compute an exposure measure $z_{ct}(g_t; w)$ (binary bands or CMA).
    \item Use BH-style recentering to obtain $\tilde{z}_{ct}$.
    \item Use $\tilde{z}_{ct}$ in 2SLS and stacked DiD to estimate dynamic ATTs.
\end{itemize}
\end{frame}

% Units & shocks
\begin{frame}{Units and Shocks}
\small
\textbf{Notation:}
\begin{itemize}
    \item $c$: Census tract, $s$: station, $t$: month or year.
    \item $P_s$: planned open date, $R_s$: realized open date.
    \item $D_s = R_s - P_s$: delay.
\end{itemize}

\medskip
\textbf{Shock vector:}
\[
g_t = \{D_s\}_s, \quad O_{st}(g_t) = \mathbf{1}\{t \ge P_s + D_s\}.
\]

\medskip
This parallels the exogenous-shock vector in \textcite{borusyak_nonrandom_2023}, where $g$ indexes the realization of common shocks affecting many units via heterogeneous exposure.
\end{frame}

% Exposure & mapping
\begin{frame}{Exposure: Walking Time and Binary Bands}
\small
Let $d^{\text{walk}}_{cs}$ be baseline (pre-period) walking time in minutes from tract $c$’s centroid to station $s$.

\medskip
\textbf{Station set within reach:}
\[
\mathcal{S}_c(T) = \{ s : d^{\text{walk}}_{cs} \le T \}, \quad T = 15 \text{ minutes.}
\]

\medskip
\textbf{Binary-band instrument:}
\begin{itemize}
    \item Bands: $\mathcal{B}_1 = [0,5]$, $\mathcal{B}_2 = (5,10]$, $\mathcal{B}_3 = (10,15]$ minutes.
\end{itemize}
\[
z^{\text{bin}}_{ct}(g_t; w) =
\sum_{s \in \mathcal{S}_c(T)} O_{st}(g_t)\Big[
\mathbf{1}\{d^{\text{walk}}_{cs} \in \mathcal{B}_1\}
+ \tfrac{2}{3}\mathbf{1}\{d^{\text{walk}}_{cs} \in \mathcal{B}_2\}
+ \tfrac{1}{3}\mathbf{1}\{d^{\text{walk}}_{cs} \in \mathcal{B}_3\}
\Big].
\]

\medskip
Weights encode decreasing influence with walk time; I will report robustness to alternative schemes and single-cutoff definitions.
\end{frame}

% CMA exposure
\begin{frame}{Alternative Exposure: Commuter Market Access (CMA)}
\small
As a continuous alternative (especially valuable with individual LODES data), I construct a commuter market access index following \citeauthor{eaton_technology_2002}, \citeauthor{donaldson_railroads_2016}, and \citeauthor{tsivanidis_evaluating_2022}'s gravity equation logic:
\[
\text{CMA}_{ct}(g_t; w) = \sum_k P_k^{\text{base}} \exp(-\kappa \, \tau_{ck}(t; g_t)),
\]
where:
\begin{itemize}
    \item $k$ indexes destination tracts,
    \item $P_k^{\text{base}}$ is baseline employment/population in tract $k$,
    \item $\tau_{ck}(t; g_t)$ is door-to-door travel time (walk + transit) from $c$ to $k$ under network $g_t$,
    \item $\kappa$ is the disutility per additional minute commuting.
\end{itemize}

\medskip
\textbf{Interpretation:} CMA summarizes how connected tract $c$ is to jobs/people in period $t$ given the realized network.
\end{frame}

% Counterfactual assignment
\begin{frame}{Counterfactual Assignment Process}
\small
To approximate $G(g_t \mid w)$, I:
\begin{itemize}
    \item Hold planned open dates $P_s$ fixed.
    \item Randomize only delays $D_s$ within “similar” station cohorts:
    \begin{itemize}
        \item Line type (light rail, tram, commuter rail)
        \item Baseline centrality class (e.g., CBD vs.\ inner vs.\ outer)
        \item Possibly metro area / corridor
    \end{itemize}
\end{itemize}

\medskip
Within each cohort $C$:
\begin{itemize}
    \item Permute the delay vector $\{D_s : s \in C\}$ to get $D_s^{(b)}$.
    \item Construct counterfactual realized dates $R_s^{(b)} = P_s + D_s^{(b)}$.
    \item Derive $O^{(b)}_{st} = \mathbf{1}\{t \ge R_s^{(b)}\}$ and corresponding exposure $z^{(b)}_{ct}$.
\end{itemize}

\medskip
This preserves geography and planned schedules but randomizes \emph{which} otherwise-similar stations were early/on-time/late.
\end{frame}

% Expected instrument & recentering
\begin{frame}{Expected Instrument and Recentering}
\small
For each tract-time $(c,t)$, simulate $B$ counterfactual schedules $g_t^{(b)}$ and compute:
\[
\mu^{\text{bin}}_{ct} = \frac{1}{B} \sum_{b=1}^B z^{\text{bin}}_{ct}(g_t^{(b)}; w),
\]
\[
\tilde{z}^{\text{bin}}_{ct} = z^{\text{bin}}_{ct}(g_t; w) - \mu^{\text{bin}}_{ct}.
\]

\medskip
\textbf{Idea:}
\begin{itemize}
    \item $\mu^{\text{bin}}_{ct}$ is the \emph{expected} exposure given baseline geography and network position.
    \item $\tilde{z}^{\text{bin}}_{ct}$ is the “as-good-as-random” component driven by realized delays.
\end{itemize}

\medskip
This recentering purges non-random exposure created by deterministic geography, leaving variation driven by shock assignment.
\end{frame}

% 2SLS
\begin{frame}{Baseline 2SLS Specification}
\small
Let $x_{ct}$ be the treatment (either $z^{\text{bin}}_{ct}$ or $\text{CMA}_{ct}$, i.e. exposure to a new station with market access gradient), and $y_{ct}$ an outcome (e.g., worker outflows or income):

\medskip
\textbf{First stage:}
\[
x_{ct} = \pi \, \tilde{z}^{\text{bin}}_{ct} + \Gamma' V_c + u_{ct},
\]
\textbf{Second stage:}
\[
y_{ct} = \beta \, \hat{x}_{ct} + \Gamma' V_c + e_{ct},
\]
where $V_c$ includes predetermined tract-level controls (and, in practice, tract and time fixed effects).

\medskip
This provides a clean starting point for using BH recentering as an instrument.
\end{frame}

% Stacked DiD contribution
\begin{frame}{My Contribution: Stacked DiD + Recentered Instrument}
\small
2SLS gives a static effect but hides:
\begin{itemize}
    \item Definition of relative ``baseline''
    \item Contributions of different adoption cohorts
\end{itemize}

\medskip
\textbf{Idea:} Embed $\tilde{z}^{\text{bin}}_{ct}$ in a stacked DiD framework \parencite{wing_stacked_2024}.

\begin{itemize}
    \item Define adoption time $A_c$ for each tract:
    \begin{itemize}
        \item Realized opening (baseline definition), or
        \item Planned opening (for anticipation / delay decompositions).
    \end{itemize}
    \item Event time $e = t - A_c$ with window $e \in [\kappa_{\text{pre}}, \kappa_{\text{post}}]$.
    \item For each adoption year $a$, form a sub-experiment:
    \begin{itemize}
        \item Treated: tracts with $A_c = a$.
        \item Controls: tracts untreated throughout the event window.
    \end{itemize}
\end{itemize}
\end{frame}

\begin{frame}{Stacked DiD + Recentered Instrument (cont.)}
\small
\textbf{Stacking and weighting:}
\begin{itemize}
    \item Stack all sub-experiments across adoption years.
    \item Apply Wing–Freedman–Hollingsworth weights so that the stacked regression recovers a trimmed aggregate ATT over $ATT(a, a+e)$.
\end{itemize}

\medskip
\textbf{Role of $\tilde{z}^{\text{bin}}_{ct}$:}
\begin{itemize}
    \item Option 1: treat $\tilde{z}^{\text{bin}}_{ct}$ as the “treatment” and interpret event-time coefficients as effects of an exogenous increase in access.
    \item Option 2: stacked IV where $x_{ct}$ (CMA or band exposure) is instrumented with $\tilde{z}^{\text{bin}}_{ct}$, and event-time indicators capture dynamics.
\end{itemize}

\medskip
This combination:
\begin{itemize}
    \item Respects the quasi-experimental timing variation from delays.
    \item Delivers an event-time interpretation of ATTs among tracts selected to receive stations.
\end{itemize}
\end{frame}

% References
\begin{frame}[noframenumbering,plain,allowframebreaks]{References}
    \printbibliography[heading=none]
\end{frame}

\end{document}
