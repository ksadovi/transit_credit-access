% Unofficial UChicago CS Poster Template
% https://github.com/k4rtik/uchicago-poster

\documentclass[final]{beamer}

% ====================
% Packages
% ====================
\usepackage[T1]{fontenc}
\usepackage{lmodern}
\usepackage[size=custom,width=120,height=72,scale=1.0]{beamerposter}
\usetheme{gemini}
\usecolortheme{uchicago}

\usepackage{graphicx}
\usepackage{booktabs}
\usepackage{doi}
\usepackage[patch=none]{microtype}
\usepackage{tikz}
\usetikzlibrary{positioning,arrows.meta}
\usepackage{pgfplots}
\pgfplotsset{compat=1.18}
\usepackage{anyfontsize}
\usepackage{amsmath,amssymb}

\pdfstringdefDisableCommands{%
\def\translate#1{#1}%
}

% ====================
% Lengths
% ====================
% (N+1)*\sepwidth + N*\colwidth = \paperwidth
\newlength{\sepwidth}
\newlength{\colwidth}
\setlength{\sepwidth}{0.025\paperwidth}
\setlength{\colwidth}{0.3\paperwidth}
\newcommand{\separatorcolumn}{\begin{column}{\sepwidth}\end{column}}

% ====================
% Title
% ====================
\title{All Aboard? Causal Evidence on Labor-Market Effects of New Rail Stations}

\author{Kyra Sadovi}

\institute[shortinst]{Harris School of Public Policy, University of Chicago}

% ====================
% Footer (optional)
% ====================
\footercontent{
  19th Winter School on Inequality and Social Welfare Theory \hfill
  \href{mailto:ksadovi@uchicago.edu}{ksadovi@uchicago.edu} \hfill
  \texttt{Draft poster: design + data construction (no results yet)}
}

% ====================
% Logo (optional)
% ====================
% If you *have* these logo files, you can uncomment and compile with them.
% Otherwise, leave commented to avoid missing-file errors.
%
% \addtobeamertemplate{headline}{}
% {
%     \begin{tikzpicture}[remember picture,overlay]
%       \node [anchor=north west, inner sep=2.5cm] at ([xshift=0.0cm,yshift=1.0cm]current page.north west)
%       {\includegraphics[height=5.0cm]{logos/uc-logo-white.eps}};
%     \end{tikzpicture}
% }

% ====================
% Body
% ====================
\begin{document}

\begin{frame}[t]
\begin{columns}[t]
\separatorcolumn

% ====================
% Column 1
% ====================
\begin{column}{\colwidth}

  \begin{block}{Motivation}
  \textbf{Question:} What is the causal effect of gaining \emph{walking-time access} to a new rail station on local labor-market outcomes?

  \vspace{0.7em}
  \textbf{Outcomes (LODES):}
  \begin{itemize}
    \item Worker flows (retention, job-to-job moves, inflows/outflows)
    \item Income proxies (where available)
  \end{itemize}

  \vspace{0.7em}
  \textbf{Mechanisms:}
  \begin{itemize}
    \item \textbf{Retention:} reliability reduces separation risk
    \item \textbf{Search/match:} expands feasible job set
    \item \textbf{Networks:} broader geographic reach increases referrals
  \end{itemize}
  \end{block}

  \begin{block}{Data + Measurement (emphasizing visuals)}
  \textbf{Transit projects and delays:} Transit Costs Project + EPA EIS timeline data.
  \begin{itemize}
    \item Planned opening $P_s$, realized opening $R_s$, delay $D_s = R_s - P_s$
  \end{itemize}

  \vspace{0.6em}
  \textbf{Walking-time exposure:} \texttt{r5r} routing on pre-period pedestrian network.
  \begin{itemize}
    \item Isochrones: 5 / 10 / 15 / 30 minute walks from tract centroids
    \item Replaces simple Euclidean buffers; incorporates barriers + street connectivity
  \end{itemize}
  \end{block}

  \begin{block}{Identification challenges}
  \begin{itemize}
    \item \textbf{Anticipation:} projects are announced years before opening $\Rightarrow$ outcomes adjust pre-treatment.
    \item \textbf{Resident sorting:} households/firms sort into connected neighborhoods.
    \item \textbf{Planner sorting:} stations placed in dense/central neighborhoods.
  \end{itemize}

  \vspace{0.7em}
  \textbf{Design choice:} focus on \emph{timing variation} from delays while acknowledging \emph{non-random exposure}.
  \end{block}

\end{column}

\separatorcolumn

% ====================
% Column 2
% ====================
\begin{column}{\colwidth}
  \begin{block}{Proof of concept: walking isochrones + worker flows}
  \begin{figure}
    \centering
    \includegraphics[width=0.98\linewidth]{3_output/2_figures/1_maps/1_station_geographies/WMATA.png}
  \end{figure}
  \begin{figure}
    \centering
    \includegraphics[width=0.98\linewidth]{3_output/dmv_map.png}
  \end{figure}
  \end{block}

  \begin{block}{Core setup (units, shocks, exposure)}
  Units are Census tracts $c$ (public LODES). Stations indexed by $s$; time by $t$ (months).

  \vspace{0.7em}
  \textbf{Station open indicator:}
  \[
    O_{st}(g_t) = \mathbf{1}\{t \ge P_s + D_s\}, \quad D_s = R_s - P_s.
  \]

  \vspace{0.6em}
  \textbf{Walking-time station set:}
  \[
    \mathcal{S}_c(T)=\{ s : d^{\text{walk}}_{cs} \le T \},\quad T=30.
  \]

  \vspace{0.6em}
  \textbf{Binary bands (baseline spec):}
  \[
  z^{\text{bin}}_{ct} = \sum_{s\in \mathcal{S}_c(T)} O_{st}\Big[
  \mathbf{1}\{d^{\text{walk}}_{cs}\in[0,5]\}
  +\tfrac{2}{3}\mathbf{1}\{d^{\text{walk}}_{cs}\in(5,10]\}
  +\tfrac{1}{3}\mathbf{1}\{d^{\text{walk}}_{cs}\in(10,15]\}\Big].
  \]
  \end{block}

  \begin{alertblock}{Key move: Borusyak--Hull recentered instrument}
  Even if \emph{timing} (delays) is quasi-random, \emph{exposure} is predictable from geography and centrality.

  \vspace{0.6em}
  \textbf{Recentering:} simulate counterfactual delay schedules (within “similar” cohorts) to compute expected exposure,
  \[
    \mu_{ct} = \frac{1}{B}\sum_{b=1}^B z^{\text{bin},(b)}_{ct},
    \qquad
    \tilde z_{ct} = z^{\text{bin}}_{ct} - \mu_{ct}.
  \]
  \textbf{Interpretation:} $\tilde z_{ct}$ isolates the “as-good-as-random” component of access driven by realized delays rather than deterministic spatial structure.
  \end{alertblock}

\end{column}

\separatorcolumn

% ====================
% Column 3
% ====================
\begin{column}{\colwidth}

  \begin{block}{Counterfactual assignment (how $\mu_{ct}$ is built)}
  \textbf{Goal:} approximate the delay assignment process while preserving planned schedules.

  \vspace{0.4em}
  \textbf{Hold fixed:} planned opening dates $P_s$.

  \textbf{Randomize:} delays $D_s$ \emph{within cohorts} that capture similarity:
  \begin{itemize}
    \item line type (light rail / tram / commuter rail)
    \item baseline centrality class (CBD / inner / outer)
    \item (optionally) corridor / metro area
  \end{itemize}

  \vspace{0.6em}
  \begin{center}
  \begin{tikzpicture}[
    node distance=1.2cm,
    box/.style={draw, rounded corners, align=center, inner sep=6pt},
    arr/.style={-Latex, thick}
  ]
    \node[box] (obs) {Observed\\delays $\{D_s\}$};
    \node[box, below=of obs] (perm) {Permute delays\\within cohorts};
    \node[box, below=of perm] (net) {Counterfactual\\open dates $R_s^{(b)}=P_s+D_s^{(b)}$};
    \node[box, below=of net] (exp) {Compute $z^{(b)}_{ct}$\\and average $\mu_{ct}$};

    \draw[arr] (obs) -- (perm);
    \draw[arr] (perm) -- (net);
    \draw[arr] (net) -- (exp);
  \end{tikzpicture}
  \end{center}
  \end{block}

  \begin{block}{Your contribution: recentering + stacked DiD for dynamics}
  \textbf{Why stack:} make cohort baselines transparent and recover event-time dynamics under staggered adoption.

  \vspace{0.6em}
  Define adoption time $A_c$ (baseline: realized opening threshold), event time $e=t-A_c$.

  \vspace{0.4em}
  For each adoption cohort $a$:
  \begin{itemize}
    \item treated: tracts with $A_c=a$
    \item controls: tracts untreated within the event window
  \end{itemize}

  \vspace{0.6em}
  \textbf{Where the BH piece enters:}
  \begin{itemize}
    \item treat $\tilde z_{ct}$ as the (exogenous) access shock inside each stacked sub-experiment, or
    \item run stacked IV with $\tilde z_{ct}$ instrumenting $x_{ct}$ (e.g., CMA / access index)
  \end{itemize}

  \vspace{0.4em}
  \textbf{Interpretation:} event-time ATTs among neighborhoods selected to receive stations (timing variation from delays).
  \end{block}

  \begin{block}{Next steps (what the audience should remember)}
  \begin{itemize}
    \item Build national station--timeline panel (EIS-based planned vs realized dates)
    \item Validate delay “as-good-as-random” within cohorts (balance / placebo tests)
    \item Estimate dynamic effects on flows and earnings proxies (LODES tract panel)
  \end{itemize}
  \end{block}

  % \begin{block}{Selected references (short)}
  % \small
  % \begin{itemize}
  %   \item Borusyak, Hull (and coauthors): recentered instruments for non-random exposure
  %   \item Wing et al.: stacked difference-in-differences
  %   \item Tsivanidis: market access / commuting-time accessibility logic
  %   \item r5r: multimodal routing + isochrone construction
  % \end{itemize}
  % \end{block}

\end{column}

\separatorcolumn
\end{columns}
\end{frame}

\end{document}
