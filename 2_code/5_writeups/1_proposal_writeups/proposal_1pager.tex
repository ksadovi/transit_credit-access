\documentclass[12pt]{article}
\usepackage{amsmath}
\usepackage{amssymb}
\usepackage{geometry}
\usepackage{setspace}
\usepackage{hyperref}
\hypersetup{
    colorlinks=true,
    linkcolor=blue,
    filecolor=black,      
    urlcolor=blue,
    pdftitle={Overleaf Example},
    pdfpagemode=FullScreen,
    citecolor=black
    }
\usepackage{titlesec}
\usepackage{ebgaramond} % Garamond font
\usepackage{titling} % Control title spacing
\usepackage[backend=biber,style=numeric]{biblatex}
\addbibresource{references.bib}


% Set page margins
\geometry{letterpaper, margin=1in}

% Single-spacing with spacing between paragraphs
\setstretch{1}
\setlength{\parskip}{.5em}

% Custom section formatting
\titleformat{\section}{\normalfont\Large\bfseries}{\thesection.}{0.5em}{}[\titlerule] % Horizontal line under section title
\titleformat{\subsection}{\normalfont\large\bfseries}{\thesubsection.}{0.5em}{} % Subsections without lines

% Title and author
\title{\textbf{Effects of Increased Access to Public Transit}}
\author{}
\date{\vspace{-2em}\today} % Minimal space for the date

\begin{document}

% Custom concise header
\noindent
\textbf{Kyra Sadovi} \hfill \textbf{\today} \\
\textbf{Effects of Increased Access to Public Transit}

% Introduction
\section{Introduction}
I want to measure the impact of increased access to public transit on nearby residents. Specifically, I want to measure the impact of physical access to a new train station on the labor market outcomes of residents surrounding the station. I will focus on two metrics: worker flows and income. 

I posit there are three mechanisms that would affect residents' labor market outcomes. First, residents are more likely to be able to retain current jobs with a more reliable mode of transportation. I am starting my analysis by measuring the impact of transit on worker flows for this reason. Second, this newfound mobility allows residents to search for higher-paying jobs in a wider selection of areas in their metro region. Third, having access to new networks that are less constrained by physical proximity will increase referral effects, i.e. the likelihood that an individual will hear about or be recommended for an employment opportunity by someone in their geographically expanded social networks. To evaluate this effectively, I will exploit the random variation of transit project construction delays to parse out public transit’s effects on these metrics from exogenous labor-market trends.

% Econometric Strategy
\section{Econometric Strategy}
Anticipatory effects arise because of the long timeline of rail transit construction projects. In McMillen and McDonald's analysis of the construction of the Orange Line in Chicago, they find “the anticipated benefits of the new transit line began to be capitalized into house prices as early as...6 years before construction was completed,”\cite{mcmillen2004reaction}. This implies that community residents, local businesses, and others adjust to the news of enhanced transit years before they gain access to it. In order to address anticipation effects, I will exploit exogenous variation in the timing of station openings generated by the unpredictable length of environmental reviews.

One other identification challenge, however, is the fact that neighborhood selection is not a random choice. In order to address this, I aim to use a difference-in-differences model which somewhat mirrors Blanco and Neri's logic for parsing out the effects of public housing demolitions on nearby residents and house prices\cite{blanco_regenerations}. While sorting into particular neighborhoods is not random, the distribution of people within a neighborhood is somewhat more random. I therefore aim to measure the difference in outcomes between residents who live within 0.25 miles of the planned station (a threshold often used in the transit literature, see Tyndall \cite{tyndall_airports}) to those who live in the same neighborhood but are more than 0.25 miles from the station. I can then compare this difference in neighborhoods who have experienced delays to those whose stations opened on time (or slightly closer to on-schedule). I will identify all workers who live within 2 miles of a train station as my sample. 
$$\text{Worker Flows}_{i,j}=\alpha + \beta\text{Open Station}_{i,j}+\gamma\text{Proximity}_i+\lambda\left(\text{Open Station}\times\text{Proximity}\right)_{i,j}+\delta_{i,j}+X_i+\varepsilon_{i,j}$$

Here, $i$ denotes the individual, $j$ denotes the time period (in terms of the index I mention in my \hyperref[sec:data]{data} section), $\text{Open Station}_{i,j}$ is a dummy for whether person $i$'s station is open in period $j$, $\delta_{i,j}$ is calendar year fixed effects, and $X_{i,j}$ is worker-level controls. $\text{Proximity}_{i}$ is a dummy for whether person $i$'s nearest station is within 0.25 miles of their home. 

% Data Sources
\section{Data Sources}
\label{sec:data}
I am using data from the LEHD Origin-Destination Employment Statistics (LODES) for the purposes of measuring the impact of public transit on worker flows. The publicly accessible version only includes data at the Census tract level. My current plan is to run preliminary regressions demonstrating my idea's feasibility using geographic units, but I am hoping to gain access to the individual-level data. It would make my idea slightly more novel (many transit papers only use geographic instead of individual data) and certainly make my results more precise. 

In order to construct the time index needed for my model, I am building a dataset of US transit projects. I begin with the \href{https://transitcosts.com/data/}{Transit Costs Project's} database of US transit projects from the last 20 years (constrained in order to best maximize LODES data use), I am collecting implementation data on each recent major public train project in the US. This information comes from Environmental Impact Statements (EIS) archived in the EPA's \href{https://cdxapps.epa.gov/cdx-enepa-II/public/action/eis/search}{EIS database}. I am logging: 
\begin{itemize}
    \item When these projects were finalized and announced (using EIS drafts as the metric for this)
    \item When (or if) these projects were successfully completed and opened to the public
    \item The intended opening date of each project
\end{itemize}
In so doing, I am creating a time index for each of these projects and tracking the construction delays on each. Such an index would equal $t=0$ when the transit project was scheduled to open; it would allow me to compare two projects, one which is open and one which is delayed, at the same timeline relative to when the projects were planned to be open to the public. Negative time periods will exist to denote the time between announcement of the station and the projected opening. 

\newpage
\printbibliography
\end{document}
