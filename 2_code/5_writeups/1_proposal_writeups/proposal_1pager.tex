\documentclass[12pt]{article}
\usepackage{amsmath}
\usepackage{amssymb}
\usepackage{geometry}
\usepackage{xcolor}
\usepackage{setspace}
\usepackage{soul}
\usepackage{hyperref}
\usepackage{graphicx}
\usepackage{float}
\usepackage{subcaption}
\hypersetup{
    colorlinks=true,
    linkcolor=blue,
    filecolor=black,      
    urlcolor=blue,
    pdftitle={Overleaf Example},
    pdfpagemode=FullScreen,
    citecolor=black
    }
\usepackage{titlesec}
\usepackage{ebgaramond} % Garamond font
\usepackage{titling} % Control title spacing
\usepackage[backend=biber,style=chicago-authordate]{biblatex}
\addbibresource{../../../references.bib}


% Set page margins
\geometry{letterpaper, margin=1in}

% Single-spacing with spacing between paragraphs
\setstretch{1}
\setlength{\parskip}{.5em}

% Custom section formatting
\titleformat{\section}{\normalfont\Large\bfseries}{\thesection.}{0.5em}{}[\titlerule] % Horizontal line under section title
\titleformat{\subsection}{\normalfont\large\bfseries}{\thesubsection.}{0.5em}{} % Subsections without lines

% Title and author
\title{\textbf{Effects of Increased Access to Public Transit}}
\author{}
\date{\vspace{-2em}\today} % Minimal space for the date

\begin{document}

% Custom concise header
\noindent
\textbf{Kyra Sadovi} \hfill \textbf{\today} \\
\textbf{Effects of Increased Access to Public Transit}

% Introduction
\section{Introduction}
I want to measure the impact of increased access to public transit on nearby residents. Specifically, I want to measure the impact of physical access to a new train station on the labor market outcomes of residents surrounding the station. I will focus on two metrics: worker flows and income. 

I posit there are three mechanisms that would affect residents' labor market outcomes. First, residents are more likely to be able to retain current jobs with a more reliable mode of transportation. I am starting my analysis by measuring the impact of transit on worker flows for this reason. Second, this newfound mobility allows residents to search for higher-paying jobs in a wider selection of areas in their metro region. Third, having access to new networks that are less constrained by physical proximity will increase referral effects, i.e. the likelihood that an individual will hear about or be recommended for an employment opportunity by someone in their geographically expanded social networks. 

To evaluate this effectively, I will exploit the as-good-as-random variation of transit project construction delays to parse out public transit’s effects on these metrics from exogenous labor-market trends. I employ \cite{borusyak_nonrandom_2023}'s recentered instrument approach to justify my treatment of delays as plausibly exogenous. 
% Model Setup
\section{Request for Advice: Incorporating Stacked Diff-in-Diff}\label{sec:advice}
I am currently thinking about how to incorporate \citeauthor{borusyak_nonrandom_2023}'s non-random exposure instrument into \citeauthor{wing_stacked_2024}'s stacked difference-in-differences approach. I think that my setup lends itself to stacked DiD. I would compare across groups of treatment time (in terms of my construction index) and within calendar year; however, I'm having trouble envisioning how I should set up my time index. 

One of the two key assumptions in \cite{wing_stacked_2024} is the absence of anticipation effects, which I initially accounted for here by comparing projects in project-time rather than calendar time. However, \citeauthor{wing_stacked_2024} require that $ATT(a,a+e)=0\quad\forall e<0$, which is the average causal effect of adopting treatment in period $a$ on outcomes experienced in $a+e$ among units in period $a$. In other words, no treatment effect before units are treated. 

While some project-time index would help with this, I think I have two paths forward: one where I define the index such that $i=0$ when a station opens, or $i=0$ when a station was \textit{expected} to open, hence my EIS approach. I contend that the latter approach combined with a dummy indicating whether a station has opened mitigates any anticipation effects; however, I wonder if the former more accurately measures anticipation effects while the latter measures the effects of delays. There's a chance that these two effects go in opposite directions -- anticipation effects encourage residents to anticipate (pardon the redundancy) future policies; excessive delays make residents less sensitive -- and less knowledgeable -- about the next period's station status. 

\begin{figure}[H]
  \centering
  % First subfigure
  \begin{subfigure}[b]{0.45\textwidth}
    \centering
    \includegraphics[width=\textwidth]{3_output/2_figures/1_maps/1_station_geographies/WMATA.png}
    \caption{Isochrones}
    \label{fig:dmv_walkingmap}
  \end{subfigure}
  \hfill
  % Second subfigure
  \begin{subfigure}[b]{0.45\textwidth}
    \centering
    \includegraphics[width=\textwidth]{3_output/dmv_map.png}
    \caption{Worker Outflows}
    \label{fig:outflows}
  \end{subfigure}
  %
  \caption{Example of what \texttt{r5r} Implementation will look like. This will become more compelling as I use change in worker flows (this is just a cross-section here).}
  \label{fig:two_side_by_side}
\end{figure}

\newpage
\printbibliography
\end{document}
