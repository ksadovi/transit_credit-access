\documentclass[12pt]{article}
\usepackage{amsmath}
\usepackage{amssymb}
\usepackage{geometry}
\usepackage{xcolor}
\usepackage{setspace}
\usepackage{soul}
\usepackage{hyperref}
\hypersetup{
    colorlinks=true,
    linkcolor=blue,
    filecolor=black,      
    urlcolor=blue,
    pdftitle={Overleaf Example},
    pdfpagemode=FullScreen,
    citecolor=black
    }
\usepackage{titlesec}
\usepackage{ebgaramond} % Garamond font
\usepackage{titling} % Control title spacing
\usepackage[backend=biber,style=chicago-authordate]{biblatex}
\addbibresource{../../../references.bib}


% Set page margins
\geometry{letterpaper, margin=1in}

% Single-spacing with spacing between paragraphs
\setstretch{1}
\setlength{\parskip}{.5em}

% Custom section formatting
\titleformat{\section}{\normalfont\Large\bfseries}{\thesection.}{0.5em}{}[\titlerule] % Horizontal line under section title
\titleformat{\subsection}{\normalfont\large\bfseries}{\thesubsection.}{0.5em}{} % Subsections without lines

% Title and author
\title{\textbf{Effects of Increased Access to Public Transit}}
\author{}
\date{\vspace{-2em}\today} % Minimal space for the date

\begin{document}

% Custom concise header
\noindent
\textbf{Kyra Sadovi} \hfill \textbf{\today} \\
\textbf{Effects of Increased Access to Public Transit}

% Introduction
\section{Introduction}
I want to measure the impact of increased access to public transit on nearby residents. Specifically, I want to measure the impact of physical access to a new train station on the labor market outcomes of residents surrounding the station. I will focus on two metrics: worker flows and income. 

I posit there are three mechanisms that would affect residents' labor market outcomes. First, residents are more likely to be able to retain current jobs with a more reliable mode of transportation. I am starting my analysis by measuring the impact of transit on worker flows for this reason. Second, this newfound mobility allows residents to search for higher-paying jobs in a wider selection of areas in their metro region. Third, having access to new networks that are less constrained by physical proximity will increase referral effects, i.e. the likelihood that an individual will hear about or be recommended for an employment opportunity by someone in their geographically expanded social networks. 

To evaluate this effectively, I will exploit the as-good-as-random variation of transit project construction delays to parse out public transit’s effects on these metrics from exogenous labor-market trends. I employ \cite{borusyak_nonrandom_2023}'s recentered instrument approach to justify my treatment of delays as plausibly exogenous. 

% Model Setup
\section{Identification Challenges}\label{sec:identification}

There are two main identification challenges in my proposed setup: anticipation effects and omitted variable bias. 

\subsection{Anticipation Effects}\label{subsec:anticipation}
Anticipatory effects arise because of the long timeline of rail transit construction projects. In \cite{mcmillen_reaction_2004}'s analysis of the construction of the Orange Line in Chicago, they find “the anticipated benefits of the new transit line began to be capitalized into house prices as early as...6 years before construction was completed.” This implies that community residents, local businesses, and others adjust to the news of enhanced transit years before they gain access to it. In order to address anticipation effects, I will exploit exogenous variation in the timing of station openings generated by the unpredictable length of environmental reviews.

\subsection{Endogenous Neighborhood Choice}\label{subsec:endog_neighborhood}
A second identification challenge is the endogeneity of neighborhood choice. Individuals and firms do not locate randomly—households with stronger labor-market attachment or higher income potential may sort into neighborhoods that are already better connected or slated for future transit investment. This sorting means that even if the timing of project completion is plausibly exogenous, the degree of exposure to new transit infrastructure is not. To address this concern, I begin with a difference-in-differences framework that mirrors \cite{blanco_regenerations}'s logic for parsing out local treatment effects while holding broader neighborhood conditions fixed. By comparing residents within the same neighborhood but at varying walking-time distances from a planned station (e.g., within 5, 10, or 15 minutes) to those beyond those thresholds, I can difference out unobserved area-level factors that jointly influence employment and investment.

\subsection{Non-Random Exposure}\label{subsec:nonrand_expo}
However, because station placement and local geography systematically shape exposure, even this within-neighborhood comparison may inherit bias from non-random exposure to the shock of a station opening. Indeed, non-random exposure is a form of sorting from the city planner side that mirrors resident sorting. Municipalities often choose to locate new transit hubs in highly populated (or highly moneyed) neighborhoods, or in central business districts. An instructive example is \cite{tyndall_airports}'s treatment of airport transit lines, another common choice for city planners. 

For this reason, I extend the design using the recentered-instrument approach of \cite{borusyak_nonrandom_2023}, which explicitly separates the random timing of openings—driven by unpredictable construction delays—from the predictable spatial pattern of exposure. In this framework, I treat delays as the exogenous shock and reweight observed accessibility by subtracting its expected value given baseline geography and network position. This correction isolates the as-good-as-random component of access growth induced by delays, allowing the resulting estimates to reflect causal effects of increased transit access rather than preexisting differences in connectivity or local composition.

I compute all walking-time distances using the R package \texttt{r5r}, which builds a multimodal transport network from OpenStreetMap and GTFS feeds. For each Census tract centroid, I generate walking isochrones of 5, 10, 15, and 30 minutes based on the pre-period pedestrian network. These isochrones are stored as polygons and spatially intersected with station coordinates to determine which stations fall within each time band. This replaces the simpler 0.25-mile Euclidean measure which I was using before, and ensures that differences in street connectivity or barriers (e.g., highways, rivers) are incorporated into my exposure measure. Using \texttt{r5r} also allows me to compute consistent door-to-door travel times $\tau_{ck}(t;g_t)$ under different network configurations for the commuter market access measure in equation (\ref{eq:CMA}). I discuss the implication of these street-network data in subsection \hyperref[subsec:r5r]{\ref{subsec:r5r}}.

\section{Model Setup}\label{sec:model_setup}
\subsection{Overview}\label{subsec:model_overview}
I extend this framework with the expected-instrument recentering approach of \cite{borusyak_nonrandom_2023}. In my setting, the shocks are construction delays in station openings, and exposure is determined by pre-period walking-time access and baseline network centrality. Recentering removes bias from non-random exposure while leveraging quasi-random timing.

\subsection{Units and Shocks}
The analysis is at the Census tract level (public LODES, see section \ref{subsec:lodes} for further discussion). Let $c$ index tracts, $s$ stations, and $t$ months. Each station has a planned open date $P_s$ and realized open date $R_s$; the delay is $D_s=R_s-P_s$. I define the shock vector $g_t=\{D_s\}_s$, which implies station status $O_{st}(g_t)=\mathbf{1}\{t\ge P_s+D_s\}$. This is equivalent to \cite{borusyak_nonrandom_2023}'s exogenous shock vector $g$.

\subsection{Exposure Variables and Mapping}
Let $d^{\text{walk}}_{cs}$ be baseline (pre-period) walking time in minutes from tract $c$'s centroid to station $s$. For a reachability horizon $T$ minutes (I use $T=30$ to start, but will be testing different horizons), define the station set
\begin{equation}
\mathcal{S}_c(T)=\{\,s:\ d^{\text{walk}}_{cs}\le T\,\}\label{eq:station_set}
\end{equation}
For now, I use binary bands within $\mathcal{S}_c(T)$ to define proximity: $\mathcal{B}_1=\{0\!-\!5\}$, $\mathcal{B}_2=\{5\!-\!10\}$, $\mathcal{B}_3=\{10\!-\!15\}$ minutes. The tract-time binary band instrument is
\begin{equation}
z^{\text{bin}}_{ct}(g_t;w)=\sum_{s\in \mathcal{S}_c(T)} O_{st}(g_t)\cdot \Big[ \mathbf{1}\{d^{\text{walk}}_{cs}\in \mathcal{B}_1\} + \tfrac{2}{3}\mathbf{1}\{d^{\text{walk}}_{cs}\in \mathcal{B}_2\} + \tfrac{1}{3}\mathbf{1}\{d^{\text{walk}}_{cs}\in \mathcal{B}_3\}\Big]\label{eq:bins}
\end{equation}
(Weights $1,\,2/3,\,1/3$ encode decreasing influence with walk time; I will report robustness to alternative schemes and to using a single 5, 10, or 15-minute cutoff.) As a continuous alternative\footnote{This alternative is what I would prefer to use given individual-level LODES data.}, I also construct a commuter market access index akin to \cite{tsivanidis_evaluating_2022}'s metric:
\begin{equation}
\text{CMA}_{ct}(g_t;w)=\sum_{k} P_k^{\text{base}}\exp(-\kappa\,\tau_{ck}(t;g_t)),\label{eq:CMA}
\end{equation}
where $\tau_{ck}(t;g_t)$ is door-to-door travel time under the network implied by $g_t$, and $P_k^{\text{base}}$ denotes baseline in destination tracts $k$. In connecting origin tract $c$ to possible destination tracts $k$, I quantify the utility of system access in tract $c$. 

\subsection{Counterfactual Assignment Process}
To approximate the shock assignment process $G(g_t\mid w)$ while preserving my planned-time indexing, I hold planned opening dates $P_s$ fixed and randomize only the delays $D_s=R_s-P_s$. Specifically, I form cohorts that capture institutional and engineering similarity but do not condition on planned year: line type (light rail, tram, commuter rail) $\times$ {baseline centrality class (measured by pre-period centrality, see discussion in section \ref{subsec:r5r}). 

Within each cohort $C$, I draw permutations $b$ of the observed delay vector $\{D_s: s\in C\}$ and assign $D_s^{(b)}=D_{\pi_C(s)}$ while keeping $P_s$ fixed, yielding counterfactual realized dates $R_s^{(b)}=P_s+D_s^{(b)}$ and counterfactual open statuses $O^{(b)}_{st}=\mathbf{1}\{t\ge R_s^{(b)}\}$. This construction preserves geography and planned schedules but randomizes \emph{which} otherwise-similar stations were early/on-time/late. 

\subsection{Expected Instrument and Recentering}
For each tract-time, I simulate $B$ counterfactual schedules $g_t^{(b)}$, recompute $z^{\text{bin}}_{ct}(g_t^{(b)};w)$ (and $\text{CMA}_{ct}(g_t^{(b)};w)$), and average:
\begin{equation}
\mu^{\text{bin}}_{ct}=\frac{1}{B}\sum_{b=1}^B z^{\text{bin}}_{ct}(g_t^{(b)};w)\label{eq:expected_instr}
\end{equation}
\begin{equation}
    \tilde{z}^{\text{bin}}_{ct}=z^{\text{bin}}_{ct}(g_t;w)-\mu^{\text{bin}}_{ct}\label{eq:recentered_z}
\end{equation}
This isolates exogenous timing-driven exposure from non-random geography/network exposure. 

\subsection{2SLS}
I estimate with two-stage least squares. Let $x_{ct}$ be the treatment (either $z^{\text{bin}}_{ct}$ itself or $\text{CMA}_{ct}$):
\begin{equation}x_{ct}=\pi\,\tilde{z}^{\text{bin}}_{ct}+\Gamma'V_{c}+u_{ct}\label{eq:1L}
\end{equation}
\begin{equation}
    y_{ct}=\beta\,\hat{x}_{ct}+\Gamma'V_{c}+e_{ct}\label{eq:2L}
\end{equation}
where $V_{ct}$ includes predetermined tract-level controls. 

% Data Sources
\section{Data Sources}
\label{sec:data}
\subsection{Worker Flow Data}\label{subsec:lodes}
I am using data from the LEHD Origin-Destination Employment Statistics (LODES) for the purposes of measuring the impact of public transit on worker flows. The publicly accessible version only includes data at the Census tract level. My current plan is to run preliminary regressions demonstrating my idea's feasibility using geographic units, but I am hoping to gain access to the individual-level data. With individual data, I would be able to calculate door-to-door walking times in equation (\ref{eq:CMA}), as well as implement the \hyperref[subsec:r5r]{R5R package} in another portion of this project. 

\subsection{Measuring Delays}\label{subsec:eis}
In order to construct the time index needed for my model, I am building a dataset of US transit projects. I begin with the \href{https://transitcosts.com/data/}{Transit Costs Project's} database of US transit projects from the last 20 years (constrained in order to best maximize LODES data use), I am collecting implementation data on each recent major public train project in the US. This information comes from Environmental Impact Statements (EIS) archived in the EPA's \href{https://cdxapps.epa.gov/cdx-enepa-II/public/action/eis/search}{EIS database}. I am logging: 
\begin{itemize}
    \item When these projects were finalized and announced (using EIS drafts as the metric for this)
    \item When (or if) these projects were successfully completed and opened to the public
    \item The intended opening date of each project
\end{itemize}
In so doing, I am creating a time index for each of these projects and tracking the construction delays on each. Such an index would equal $t=0$ when the transit project was scheduled to open; it would allow me to compare two projects, one which is open and one which is delayed, at the same timeline relative to when the projects were planned to be open to the public. Negative time periods will exist to denote the time between announcement of the station and the projected opening. 

\subsection{Walking Time}\label{subsec:r5r}
I use the \texttt{r5r} R package to compute tract-level walking times and travel-time matrices \parencite{r5r}. \texttt{r5r} constructs a routing graph that integrates pedestrian, bus, and rail modes using GTFS feeds and OpenStreetMap data. I have loaded the pedestrian street maps of each transit system I am measuring. I plan to use only the pre-period pedestrian layer so that accessibility is not contaminated by street or station-area improvements that occur after project completion. That way I can measure walking times between each tract centroid and its closest station (or set of stations $S$, as defined above).

For each tract centroid, I calculate walking-time isochrones at 5, 10, 15, and 30 minutes. These polygons define the bands $\mathcal{B}_1$–$\mathcal{B}_3$ used in the binary-band instrument, while the 30-minute horizon $\mathcal{S}_c(30)$ defines the overall reachability set of potential stations. The resulting tract–station matrix $d^{\text{walk}}_{cs}$ feeds directly into the construction of $z^{\text{bin}}_{ct}$ and the CMA measure in equation (\ref{eq:CMA}). I store all travel-time matrices and isochrones as \texttt{sf} objects to facilitate spatial joins and to ensure that exposure variables are generated reproducibly across network scenarios.

In my next steps, I can leverage \texttt{r5r::travel\_time\_matrix()} to compute $\tau_{ck}(t;g_t)$ across origin–destination pairs for both realized and counterfactual network configurations. This step allows the CMA-based specification to reflect how the effective job-access radius expands or contracts under each simulated delay schedule.

The level of granularity achievable through this package underscores why getting access to individual LODES data will be useful for this project. I would be able to measure CMA continuously with higher confidence, whereas relying on tract centroids at the moment makes it more defensible to use the binning strategy in equation (\hyperref[eq:bins]{\ref{eq:bins}}).

\newpage
\printbibliography
\end{document}
