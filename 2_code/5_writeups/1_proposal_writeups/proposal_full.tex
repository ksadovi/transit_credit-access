\documentclass[12pt]{article}
\usepackage{amsmath}
\usepackage{amssymb}
\usepackage{geometry}
\usepackage{xcolor}
\usepackage{setspace}
\usepackage{soul}
\usepackage{hyperref}
\usepackage[affil-it]{authblk}
\usepackage{graphicx}
\usepackage{float}
\usepackage{subcaption}
\hypersetup{
    colorlinks=true,
    linkcolor=blue,
    filecolor=black,      
    urlcolor=blue,
    pdftitle={Overleaf Example},
    pdfpagemode=FullScreen,
    citecolor=black
    }
\usepackage{titlesec}
\usepackage{titling} % Control title spacing
\usepackage[backend=biber,style=chicago-authordate]{biblatex}
\addbibresource{../../../references.bib}


% Set page margins
\geometry{letterpaper, margin=1in}

% Single-spacing with spacing between paragraphs
\setstretch{1}
\setlength{\parskip}{.5em}

% Custom section formatting
\titleformat{\section}{\normalfont\Large\bfseries}{\thesection.}{0.5em}{}
\titleformat{\subsection}{\normalfont\large\bfseries}{\thesubsection.}{0.5em}{} % Subsections without lines

% Title and author
\title{All Aboard? Causal Evidence on Labor-Market Effects of New Rail Stations}
\author{Kyra Sadovi
  \thanks{E-Mail Address: \texttt{ksadovi@uchicago.edu}}}
\affil{Harris School of Public Policy,\\ University of Chicago\\}
\date{\vspace{-2em}\today} % Minimal space for the date

\begin{document}
\makeatletter
\def\@maketitle{%
  \newpage
  \null
  \vskip 2em%
  \begin{center}%
  \let \footnote \thanks
    {\Large\bfseries \@title \par}%
    \vskip 1.5em%
    {\normalsize
      \lineskip .5em%
      \begin{tabular}[t]{c}%
        \@author
      \end{tabular}\par}%
    \vskip 1em%
    {\normalsize \@date}%
  \end{center}%
  \par
  \vskip 1.5em}
\makeatother
\begin{titlepage}
\maketitle
\begin{abstract}
    I study how improved access to public transit affects nearby residents’ labor-market outcomes, focusing on worker flows and income. Increased access to a new rail station may help workers retain existing jobs through more reliable commuting, expand their ability to search for higher-paying opportunities, and widen referral networks by connecting them to a larger set of locations. To identify these effects, I exploit quasi-random variation generated by construction delays in transit projects and apply the recentered-instrument approach of \cite{borusyak_nonrandom_2023} to isolate exogenous changes in access from underlying labor-market trends. I then combine this recentered instrument approach with stacked difference-in-differences to develop a novel approach to grappling with spatial endogeneity over time. 
\end{abstract}
\end{titlepage}


% Introduction
\section{Introduction}
I want to measure the impact of increased access to public transit on nearby residents. Specifically, I want to measure the impact of physical access to a new train station on the labor market outcomes of residents surrounding the station. I will focus on two metrics: worker flows and income. 

I posit there are three mechanisms that would affect residents' labor market outcomes. First, residents are more likely to be able to retain current jobs with a more reliable mode of transportation. I am starting my analysis by measuring the impact of transit on worker flows for this reason. Second, this newfound mobility allows residents to search for higher-paying jobs in a wider selection of areas in their metro region. Third, having access to new networks that are less constrained by physical proximity will increase referral effects, i.e. the likelihood that an individual will hear about or be recommended for an employment opportunity by someone in their geographically expanded social networks. 

To evaluate this effectively, I will exploit the as-good-as-random variation of transit project construction delays to parse out public transit’s effects on these metrics from exogenous labor-market trends. I employ \cite{borusyak_nonrandom_2023}'s recentered instrument approach to justify my treatment of delays as plausibly exogenous. 

% Model Setup
\section{Identification Challenges}\label{sec:identification}
There are three main identification challenges in my proposed setup: anticipation effects, endogeneity of resident neighborhood choice, and endogeneity of station location choice. 

\subsection{Anticipation Effects}\label{subsec:anticipation}
Anticipatory effects arise because of the long timeline of rail transit construction projects. In \cite{mcmillen_reaction_2004}'s analysis of the construction of the Orange Line in Chicago, they find ``the anticipated benefits of the new transit line began to be capitalized into house prices as early as\ldots6 years before construction was completed.'' This implies that community residents, local businesses, and other stakeholders adjust to the news of enhanced transit years before they gain access to it. In order to address anticipation effects, I will exploit exogenous variation in the timing of station openings generated by the unpredictable length of environmental reviews, as I discuss in section \ref{subsec:eis}.

\subsection{Non-Random Exposure}\label{subsec:nonrand_expo}
The last identification challenge is the endogeneity of neighborhood choice. Individuals and firms do not locate randomly—households with stronger labor-market attachment or higher income potential may sort into neighborhoods that are already better connected or slated for future transit investment. This sorting means that even if the timing of project completion is plausibly exogenous, the degree of exposure to new transit infrastructure is not. 

For this reason, I extend the design using the recentered-instrument approach of \cite{borusyak_nonrandom_2023}, which explicitly separates the random timing of openings—driven by unpredictable construction delays—from the predictable spatial pattern of exposure. In this framework, I treat delays as an exogenous shock and reweight observed accessibility by subtracting its expected value given baseline geography and network position. This correction isolates the as-good-as-random component of access growth induced by delays, allowing the resulting estimates to reflect causal effects of increased transit access rather than preexisting differences in connectivity or local composition.

I compute all walking-time distances using the R package \texttt{r5r}, which builds a multimodal transport network from OpenStreetMap and GTFS feeds. For each Census tract centroid, I generate walking isochrones of 5, 10, 15, and 30 minutes based on the pre-period pedestrian network. These isochrones are stored as polygons and spatially intersected with station coordinates to determine which stations fall within each time band. This replaces the simpler 0.25-mile Euclidean measure which I was using before, and ensures that differences in street connectivity or barriers (e.g., highways, rivers) are incorporated into my exposure measure. Using \texttt{r5r} also allows me to compute consistent door-to-door travel times $\tau_{ck}(t;g_t)$ under different network configurations for the commuter market access measure in equation (\ref{eq:CMA}). I discuss the implication of these street-network data in subsection \hyperref[subsec:r5r]{\ref{subsec:r5r}}.

\subsection{Endogenous Station Location Choice Choice}\label{subsec:endog_neighborhood}
However, because station placement and local geography systematically shape exposure, even this within-neighborhood comparison may inherit bias from non-random exposure to the treatment/shock of a station opening. Indeed, non-random exposure is a form of sorting from the city planner side that mirrors resident sorting. Municipalities often choose to locate new transit hubs in highly populated (or highly moneyed) neighborhoods, or in central business districts. An instructive example is \cite{tyndall_airports}'s treatment of airport transit lines, another common choice for city planners. 

This problem is addressed twofold. I solve it implicitly by working only with neighborhoods that have already been selected for a transit station. In varying only the opening date of the station rather than the neighborhood selection for each station, I am implicitly identifying the ATT as my causal parameter of interest. 

The second potential solution is to categorize each neighborhood by type --- as seen in \cite{tyndall_airports}, different station locations are chosen for different reasons. I could manually code each station location with the intended use for it (commuting destination, residential location, retail hub, airport, etc.) and control for these types. However, doing this would narrow the number of comparisons I could make substantially. 

\section{Model Setup}\label{sec:model_setup}
\subsection{Overview}\label{subsec:model_overview}
I extend this framework with the expected-instrument recentering approach of \cite{borusyak_nonrandom_2023}. In my setting, the shocks are construction delays in station openings, and exposure is determined by pre-period walking-time access and baseline network centrality. Recentering removes bias from non-random exposure while leveraging quasi-random timing.

\subsection{Units and Shocks}
The analysis is at the Census tract level (public LODES, see section \ref{subsec:lodes} for further discussion). Let $c$ index tracts, $s$ stations, and $t$ months. Each station has a planned open date $P_s$ and realized open date $R_s$; the delay is $D_s=R_s-P_s$. I define the shock vector $g_t=\{D_s\}_s$, which implies station status $O_{st}(g_t)=\mathbf{1}\{t\ge P_s+D_s\}$. This is equivalent to \cite{borusyak_nonrandom_2023}'s exogenous shock vector $g$.

\subsection{Exposure Variables and Mapping}
Let $d^{\text{walk}}_{cs}$ be baseline (pre-period) walking time in minutes from tract $c$'s centroid to station $s$. For a reachability horizon $T$ minutes (I use $T=30$ to start, but will be testing different horizons), define the station set
\begin{equation}
\mathcal{S}_c(T)=\{\,s:\ d^{\text{walk}}_{cs}\le T\,\}\label{eq:station_set}
\end{equation}
For now, I use binary bands within $\mathcal{S}_c(T)$ to define proximity: $\mathcal{B}_1=\{0\!-\!5\}$, $\mathcal{B}_2=\{5\!-\!10\}$, $\mathcal{B}_3=\{10\!-\!15\}$ minutes. The tract-time binary band instrument is
\begin{equation}
z^{\text{bin}}_{ct}(g_t;w)=\sum_{s\in \mathcal{S}_c(T)} O_{st}(g_t)\cdot \Big[ \mathbf{1}\{d^{\text{walk}}_{cs}\in \mathcal{B}_1\} + \tfrac{2}{3}\mathbf{1}\{d^{\text{walk}}_{cs}\in \mathcal{B}_2\} + \tfrac{1}{3}\mathbf{1}\{d^{\text{walk}}_{cs}\in \mathcal{B}_3\}\Big]\label{eq:bins}
\end{equation}
(Weights $1,\,2/3,\,1/3$ encode decreasing influence with walk time; I will report robustness to alternative schemes and to using a single 5, 10, or 15-minute cutoff.) As a continuous alternative\footnote{This alternative is what I would prefer to use given individual-level LODES data.}, I also construct a commuter market access index akin to \cite{tsivanidis_evaluating_2022}'s metric:
\begin{equation}
\text{CMA}_{ct}(g_t;w)=\sum_{k} P_k^{\text{base}}\exp(-\kappa\,\tau_{ck}(t;g_t)),\label{eq:CMA}
\end{equation}
where $\tau_{ck}(t;g_t)$ is door-to-door travel time under the network implied by $g_t$, and $P_k^{\text{base}}$ denotes baseline in destination tracts $k$. In connecting origin tract $c$ to possible destination tracts $k$, I quantify the utility of system access in tract $c$. These expected market-access measures are equivalent to \citeauthor{borusyak_nonrandom_2023}'s expected exposure vector $w$. 

\subsection{Counterfactual Assignment Process}
To approximate the shock assignment process $G(g_t\mid w)$ while preserving my planned-time indexing, I hold planned opening dates $P_s$ fixed and randomize only the delays $D_s=R_s-P_s$. Specifically, I form cohorts that capture institutional and engineering similarity but do not condition on planned year: line type (light rail, tram, commuter rail) $\times$ {baseline centrality class (measured by pre-period centrality, see discussion in section \ref{subsec:r5r}). 

Within each cohort $C$, I draw permutations $b$ of the observed delay vector $\{D_s: s\in C\}$ and assign $D_s^{(b)}=D_{\pi_C(s)}$ while keeping $P_s$ fixed, yielding counterfactual realized dates $R_s^{(b)}=P_s+D_s^{(b)}$ and counterfactual open statuses $O^{(b)}_{st}=\mathbf{1}\{t\ge R_s^{(b)}\}$. This construction preserves geography and planned schedules but randomizes \emph{which} otherwise-similar stations were early/on-time/late. 

\subsection{Expected Instrument and Recentering}
For each tract-time, I simulate $B$ counterfactual schedules $g_t^{(b)}$, recompute $z^{\text{bin}}_{ct}(g_t^{(b)};w)$ (and $\text{CMA}_{ct}(g_t^{(b)};w)$), and average:
\begin{equation}
\mu^{\text{bin}}_{ct}=\frac{1}{B}\sum_{b=1}^B z^{\text{bin}}_{ct}(g_t^{(b)};w)\label{eq:expected_instr}
\end{equation}
\begin{equation}
    \tilde{z}^{\text{bin}}_{ct}=z^{\text{bin}}_{ct}(g_t;w)-\mu^{\text{bin}}_{ct}\label{eq:recentered_z}
\end{equation}
This isolates exogenous timing-driven exposure from non-random geography/network exposure. 

\subsection{My Contribution: Stacked DiD + \citeauthor{borusyak_nonrandom_2023}}
While the instrument specification above provides a straightforward way to use the recentered instrument $\tilde{z}^{\text{bin}}_{ct}$, it does not make explicit how treatment effects evolve over time or how different cohorts contribute to the overall estimate. To study dynamic effects and align more directly with recent work on staggered adoption, I am exploring an alternative formulation that embeds the Borusyak–Hull recentered instrument into a stacked difference-in-differences (DiD) framework in the spirit of \textcite{wing_stacked_2024}.

In this setting, each tract $c$ has an \emph{adoption time} $A_c$ defined as the period in which its nearest station crosses a treatment threshold. I can define $A_c$ using either the \emph{realized} opening date of the station (when access actually improves) or the \emph{planned} opening date from the EIS (when residents may begin to adjust in anticipation). Realized-time cohorts are more natural for enforcing the no-anticipation condition $ATT(a,a+e)=0$ for all $e<0$, while planned-time cohorts are better suited to detecting anticipatory behavior around expected opening dates. In practice, I plan to take realized opening as the baseline definition of $A_c$ and use the planned-time index in robustness exercises that explicitly examine anticipation and delay effects.

Given an adoption time $A_c$, I define event time $e=t-A_c$ and focus on an event window $e \in [\kappa_{\text{pre}},\kappa_{\text{post}}]$ (for example, three years before to five years after adoption). For each adoption cohort $a$ within this trimmed window, I form a sub-experiment that compares tracts first treated in year $a$ to tracts that remain untreated throughout the corresponding event window. Stacking these sub-experiments yields a balanced event-study design in which changes in the estimated effects across $e$ reflect dynamics rather than changes in the underlying composition of cohorts. Following \textcite{wing_stacked_2024}, I can apply corrective weights so that the stacked regression recovers a trimmed aggregate average treatment effect on the treated, aggregating the group-time effects $ATT(a,a+e)$ into a well-defined event-time parameter.

The recentered instrument enters this framework by purging non-random exposure within each sub-experiment before aggregation. One option is to treat $\tilde{z}^{\text{bin}}_{ct}$ as the treatment variable in the stacked DiD and interpret the coefficients on event-time indicators as capturing the causal effect of an exogenous increase in access driven by delays. A closely related alternative is to run a stacked IV specification in which the endogenous treatment $x_{ct}$ (e.g., CMA or the binary-band exposure) is instrumented with $\tilde{z}^{\text{bin}}_{ct}$ within the stacked panel, while adoption-cohort-by-event-time indicators capture dynamics. In both cases, the combination of Borusyak–Hull recentering and weighted stacked DiD respects the quasi-experimental timing variation generated by construction delays and yields a transparent, event-time interpretation of the average treatment effect among tracts that were selected to receive a station.

% \textbf{Add portion about stacking by year cohorts, then permuting within those cohorts \textit{when} areas are treated. Don't want to do exactly \citeauthor{borusyak_nonrandom_2023}'s thing because they care about nonrandom choice of station location; I care about accounting for nonrandom delays because I'm comparing within people who have been chosen (make explicit that I'm identifying the causal ATT).}

% I am currently thinking about how to incorporate \citeauthor{borusyak_nonrandom_2023}'s non-random exposure instrument into \citeauthor{wing_stacked_2024}'s stacked difference-in-differences approach. I think that my setup lends itself to stacked DiD. I would compare across groups of treatment time (in terms of my construction index) and within calendar year; however, I'm having trouble envisioning how I should set up my time index. 

% One of the two key assumptions in \cite{wing_stacked_2024} is the absence of anticipation effects, which I initially accounted for here by comparing projects in project-time rather than calendar time. However, \citeauthor{wing_stacked_2024} require that $ATT(a,a+e)=0\quad\forall e<0$, which is the average causal effect of adopting treatment in period $a$ on outcomes experienced in $a+e$ among units in period $a$. In other words, no treatment effect before units are treated. 

% While some project-time index would help with this, I think I have two paths forward: one where I define the index such that $i=0$ when a station opens, or $i=0$ when a station was \textit{expected} to open, hence my EIS approach. I contend that the latter approach combined with a dummy indicating whether a station has opened mitigates any anticipation effects; however, I wonder if the former more accurately measures anticipation effects while the latter measures the effects of delays. There's a chance that these two effects go in opposite directions -- anticipation effects encourage residents to anticipate (pardon the redundancy) future policies; excessive delays make residents less sensitive -- and less knowledgeable -- about the next period's station status. 

% Data Sources
\section{Data Sources}
\label{sec:data}
\subsection{Worker Flow Data}\label{subsec:lodes}
I am using data from the LEHD Origin-Destination Employment Statistics (LODES) for the purposes of measuring the impact of public transit on worker flows. The publicly accessible version only includes data at the Census tract level. My current plan is to run preliminary regressions demonstrating my idea's feasibility using geographic units, but I am hoping to gain access to the individual-level data. With individual data, I would be able to calculate door-to-door walking times in equation (\ref{eq:CMA}), as well as implement the \hyperref[subsec:r5r]{R5R package} in another portion of this project. 

\subsection{Measuring Delays}\label{subsec:eis}
In order to construct the time index needed for my model, I am building a dataset of US transit projects. I begin with the \href{https://transitcosts.com/data/}{Transit Costs Project's} database of US transit projects from the last 20 years (constrained in order to best maximize LODES data use), I am collecting implementation data on each recent major public train project in the US. This information comes from Environmental Impact Statements (EIS) archived in the EPA's \href{https://cdxapps.epa.gov/cdx-enepa-II/public/action/eis/search}{EIS database}. I am logging: 
\begin{itemize}
    \item When these projects were finalized and announced (using EIS drafts as the metric for this)
    \item When (or if) these projects were successfully completed and opened to the public
    \item The intended opening date of each project
\end{itemize}
In so doing, I am creating a time index for each of these projects and tracking the construction delays on each. Such an index would equal $t=0$ when the transit project was scheduled to open; it would allow me to compare two projects, one which is open and one which is delayed, at the same timeline relative to when the projects were planned to be open to the public. Negative time periods will exist to denote the time between announcement of the station and the projected opening. 

\subsection{Walking Time}\label{subsec:r5r}
I use the \texttt{r5r} R package to compute tract-level walking times and travel-time matrices \parencite{r5r}. \texttt{r5r} constructs a routing graph that integrates pedestrian, bus, and rail modes using GTFS feeds and OpenStreetMap data. I have loaded the pedestrian street maps of each transit system I am measuring. I plan to use only the pre-period pedestrian layer so that accessibility is not contaminated by street or station-area improvements that occur after project completion. That way I can measure walking times between each tract centroid and its closest station (or set of stations $S$, as defined above).

For each tract centroid, I calculate walking-time isochrones at 5, 10, 15, and 30 minutes. These polygons define the bands $\mathcal{B}_1$–$\mathcal{B}_3$ used in the binary-band instrument, while the 30-minute horizon $\mathcal{S}_c(30)$ defines the overall reachability set of potential stations. The resulting tract–station matrix $d^{\text{walk}}_{cs}$ feeds directly into the construction of $z^{\text{bin}}_{ct}$ and the CMA measure in equation (\ref{eq:CMA}). I store all travel-time matrices and isochrones as \texttt{sf} objects to facilitate spatial joins and to ensure that exposure variables are generated reproducibly across network scenarios.

In my next steps, I can leverage \texttt{r5r::travel\_time\_matrix()} to compute $\tau_{ck}(t;g_t)$ across origin–destination pairs for both realized and counterfactual network configurations. This step allows the CMA-based specification to reflect how the effective job-access radius expands or contracts under each simulated delay schedule.

The level of granularity achievable through this package underscores why getting access to individual LODES data will be useful for this project. I would be able to measure CMA continuously with higher confidence, whereas relying on tract centroids at the moment makes it more defensible to use the binning strategy in equation (\hyperref[eq:bins]{\ref{eq:bins}}).

\begin{figure}[H]
  \centering
  % First subfigure
  \begin{subfigure}[b]{0.45\textwidth}
    \centering
    \includegraphics[width=\textwidth]{3_output/2_figures/1_maps/1_station_geographies/WMATA.png}
    \caption{Isochrones}
    \label{fig:dmv_walkingmap}
  \end{subfigure}
  \hfill
  % Second subfigure
  \begin{subfigure}[b]{0.45\textwidth}
    \centering
    \includegraphics[width=\textwidth]{3_output/dmv_map.png}
    \caption{Worker Outflows}
    \label{fig:outflows}
  \end{subfigure}
  %
  \caption{Example of what \texttt{r5r} Implementation will look like. This will become more compelling as I use change in worker flows (this is just a cross-section here).}
  \label{fig:two_side_by_side}
\end{figure}

\newpage
\printbibliography
\end{document}
