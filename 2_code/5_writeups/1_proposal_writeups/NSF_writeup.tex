\documentclass[12pt]{article}
\usepackage{amsmath}
\usepackage{geometry}
\geometry{letterpaper, margin=1in}
\usepackage{setspace}
\usepackage{enumitem}

\title{Graduate Research Plan Statement}
\author{Kyra Sadovi}
\date{}

\begin{document}

\maketitle
\section*{Summary}
Public transportation is a public good used disproportionately by low-income individuals. Increasing access to public transit could positively impact low-income populations, and research on the topic could inform urban policymakers on future infrastructure plans. I will take advantage of new train station openings over the past two decades to measure the effect of increased access to public transportation on wages. I posit two mechanisms that may increase residents’ wages. First, residents are more likely to be able to retain current jobs with a more reliable mode of transportation. Second, this newfound mobility allows residents to search for higher-paying jobs in a wider selection of areas in their metro region. To evaluate this effectively, I will exploit the random variation of transit project construction delays to parse out transit’s effects on wages from exogenous labor-market trends.

\section*{Empirical Strategy}
Isolating the effect of increased access to public transit on wages is complicated by two confounding dynamics: changing neighborhood compositions and anticipatory effects. I will address these two issues respectively by:
\begin{enumerate}
    \item Measuring the effect of increased transit on individuals rather than neighborhoods, and
    \item Exploiting quasi-random variation in construction delays.
\end{enumerate}

\subsection*{Addressing Composition Effects}
Rail stations are a highly valued amenity in many urban neighborhoods. A new rail station is likely to cause an influx of residents as individuals who value transit are sorted into the area. However, this migration will change the composition of the neighborhood – for example, if individuals with the means to live in a sought-after area move in, incumbent residents may be priced out. A comparison of neighborhoods at a spatial level before and after transit access is confounded by this shift in composition. I will therefore track individuals, rather than neighborhoods, over time to isolate the individual-level effect.

\subsection*{Addressing Anticipatory Effects}
Anticipatory effects arise because of the long timeline of rail transit construction projects. For example, McMillen and McDonald demonstrate that “the anticipated benefits of the new transit line began to be capitalized into house prices as early as 1987, 6 years before construction was completed”\cite{mcmillen2004reaction}. This implies that community residents, local businesses, and others adjust to the news of enhanced transit years before they gain access to it. I will exploit as-good-as-random variation in the timing of station openings generated by the unpredictable length of environmental reviews, using a difference-in-differences estimator to compare wage growth of residents near newly opened stations to those near a station whose opening is still delayed by review.

\section*{Data and Analysis}
I will employ the Longitudinal Employer-Household Dynamics (LEHD) data at the Census Bureau to track the residential location and labor market outcomes of individuals over time. I will also use the Transit Costs Project (TCP) database to identify 21 U.S.-based rail projects featuring 187 station openings across 10 cities in the last 20 years. Using difference-in-differences, I will regress wages of individual \(i\) during period \(j\) on the station status \(OpenStation_{i,j}\), controlling for neighborhood fixed effects \(\theta_k\), period fixed effects \(\gamma_j\), and a vector of worker-level controls \(X_i\):
\[
Wages_{i,j} = \beta OpenStation_{i,j} + \theta_k + \gamma_j + X_i + \epsilon_{i,j}.
\]

\section*{Intellectual Merit}
Previous papers have focused on the impact of transit on neighborhood-wide employment, but the effect of transit on individual welfare is studied significantly less frequently. Neighborhood-level measures are confounded by changing compositions in response to transit access. My analysis will allow me to measure individual-level effects and address biases in previous studies that measure compositional rather than individual effects.

\section*{Broader Impacts}
Public transit riders are disproportionately low-income compared to the general population. However, transit is not universally available. Nearly half (45\%) of Americans lack access to public transportation, and 41\% of households have one or fewer vehicles. Expanding transit could increase commuting and employment opportunities, especially for lower-income households. This research will provide policymakers with a crucial benefit component of cost-benefit analyses of new transit projects, helping them prioritize locations for maximal community benefit.

\section*{References}
\begin{enumerate}[label={[}\arabic*{]}]
    \item McMillen, D.P. and McDonald, J. (2004). Reaction of House Prices to a New Rapid Transit Line: Chicago's Midway Line, 1983–1999.
    \item Tyndall, J. (2019). The Local Labour Market Effects of Light Rail Transit.
    \item U.S. Census Bureau report “Commuting by Public Transportation in the United States: 2019”.
    \item ASCE report “Infrastructure Report Card: Transit”.
\end{enumerate}

\end{document}
